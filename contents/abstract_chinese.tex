% !TEX root = ../main.tex

% 定义中文摘要和关键字
\begin{cabstract}
请注意,以下内容主要参考自薛瑞尼的清华大学论文模板。

论文的摘要是对论文研究内容和成果的高度概括。
摘要应对论文所研究的问题及其研究\textbf{目的}进行描述,
对研究\textbf{方法和过程}进行简单介绍,
对研究\textbf{成果和所得结论}进行概括。
摘要应具有独立性和自明性,
其内容应包含与论文全文等量的主要信息。
使读者即使不阅读全文,
通过摘要就能了解论文的总体内容和主要成果。

论文摘要的书写应力求精确、简明。切忌写成对论文书写内容进行提要的形式,尤其要避
免“第 1 章……;第 2 章……;……”这种或类似的陈述方式。
不宜使用公式、图表,不标注引用文献。
硕士论文摘要的字数一般为300--500 个左右。

关键词是为了文献标引工作、
用以表示全文主要内容信息的单词或术语。
关键词不超过 5个,
每个关键词中间用分号分隔
(研究生院要求使用分号分隔,软件学院要求使用逗号分隔)。
见源码\texttt{zjuthesis.cls}搜索keywords了解。
\end{cabstract}

\ckeywords{\LaTeX, CJK, 模板, 毕业论文}
