% !Mode:: "TeX:UTF-8"
% !TEX builder = LATEXMK
% !TEX program = xelatex
\documentclass[master,oneside]{zjuthesis} % 如果你的论文不满80页,还是单面印刷吧


%%%%%%%%%%%%%%%%%%%%%%%%%%%%%% 开始填写前置部分使用的变量
%%%%%%%%%%%%%%%%%%%%%%%%%%%%%% 样式设定在 zjuthesis.cls 下, 人类可读,爱请查阅

% 这里写这么鬼畜是为了测试多几个字会不会造成溢出
\title{某某某某某科学的浙江大学软件学院硕士论文模板} % 封面和题名页使用
\englishtitle{Toaru Kagaku no Template of Software Engineering Thesis in \LaTeX } % 封面和题名页使用
% 如果您的标题用字过多,请自行调节 zjuthesis.cls 里的 ZJUmakecover 里的各项距离。

%\author{伊藤春希}          % 申请人姓名 封面使用
\author{君之名}

\classification{TP311.1}    % 封面头使用
\serialnumber{10335}        % 封面头使用
\secretlevel{无}            % 封面头使用
\studentnumber{21x51xxx}    % 封面头使用

%\supervisor{御坂美琴}       % 导师 封面使用
\supervisor{师之名}
%\spvtitle{电击使}           % 职称 封面使用
\spvtitle{职称}              % 职称 封面使用

%\cpsupervisor{桂和纱}       % 合作导师,如果没有合作导师,就在此文件第 4 行\documentclass选项栏中加上"nocpsupervisor"。
\cpsupervisor{师之名}
\cspvtitle{职称}             % 合作导师职称

% 从机械工程学院改来,保留设定变量命名
%\major{船舶工程}            % 专业学位类别栏 填 工程硕士
\major{工程硕士}
%\research{白学}             % 专业学位领域栏 填 软件工程
\research{软件工程}
\institute{软件学院}         % 所在学位栏 填 软件学院

\submitdate{2333年2月33日}   % 论文提交日期 栏

% 题名页的评阅人及答辩席
% 归档时候填写
% 论文评阅人1 2 3 4 5
\reviewerA{} \enreviewerA{}
\reviewerB{} \enreviewerB{}
\reviewerC{} \enreviewerC{}
\reviewerD{} \enreviewerD{}
\reviewerE{} \enreviewerE{}

% 答辩委员会主席
\chairperson{} \enchairperson{}

% 答辩委员 1 2 3 4 5
\commissionerA{} \encommissionerA{}
\commissionerB{} \encommissionerB{}
\commissionerC{} \encommissionerC{}
\commissionerD{} \encommissionerD{}
\commissionerE{} \encommissionerE{}

% 答辩日期
\defencedate{} \eendefencedate{}  % 因为endefencedate 命名被占用

% 论文前置部分变量填写完毕 开始全书排版
\begin{document}

% 封面、中文题名页、英文题名页、独创声明和版权使用书 无页码
\maketitle

% 摘要部分
\abstractmatter
% !TEX root = ../main.tex

% 定义中文摘要和关键字
\begin{cabstract}
请注意,以下内容主要参考自薛瑞尼的清华大学论文模板。

论文的摘要是对论文研究内容和成果的高度概括。
摘要应对论文所研究的问题及其研究\textbf{目的}进行描述,
对研究\textbf{方法和过程}进行简单介绍,
对研究\textbf{成果和所得结论}进行概括。
摘要应具有独立性和自明性,
其内容应包含与论文全文等量的主要信息。
使读者即使不阅读全文,
通过摘要就能了解论文的总体内容和主要成果。

论文摘要的书写应力求精确、简明。切忌写成对论文书写内容进行提要的形式,尤其要避
免“第 1 章……;第 2 章……;……”这种或类似的陈述方式。
不宜使用公式、图表,不标注引用文献。
硕士论文摘要的字数一般为300--500 个左右。

关键词是为了文献标引工作、
用以表示全文主要内容信息的单词或术语。
关键词不超过 5个,
每个关键词中间用分号分隔
(研究生院要求使用分号分隔,软件学院要求使用逗号分隔)。
见源码\texttt{zjuthesis.cls}搜索keywords了解。
\end{cabstract}

\ckeywords{\LaTeX, CJK, 模板, 毕业论文}

\include{contents/abstract_english}

% 目录和术语表
\frontmatter
\tableofcontents % 正文目录
\listoffigures   % 图目录
\listoftables    % 表目录
% 术语及缩略词表(需要则开)
%\include{contents/denotation}

% 正文排版开始 建议一章一文件 (好像无法嵌套 include) 
\mainmatter
% !TEX root = ../main.tex

% 第一章一般名为绪论/引言,不可省略

\chapter{绪论}

\section{研究背景}

绪论第一节一般是研究背景,
交待下这个领域遇到了什么亟待改善的困境。
从严谨的学术观念考虑,必须引用足量的数据描述现状,
避免使用过多主观判断的语句和用词。

\section{国内外研究现状}

对于国内硕士学位论文来说,
一般较少研究完全无前人探索的领域,
所以有必要交待前人在此做出的努力和尝试。
同样,请提供数据和引用保证严谨。

为避免引起评阅老师判定有凑篇幅之嫌,
请有针对的描述前人研究的不足之处,
做到``有破有立''。

\section{论文研究内容}

此部分必须详细描述,
必要时可划设小节。
国外学位论文的Introduction章基本仅阐述此内容。
为研究开展的相关工作和实验,
此间遇到何难处及对应的解法。
对论文研究领域不甚了解的评阅老师,
希望从摘要和此小节尽可能的了解最多信息。


\section{论文组织结构}

简明扼要的介绍下各章主旨,版面控制半页内。
 % 绪论
%!TEX root = ../main.tex

\chapter{为何使用\LaTeX 排版学位论文\footnote{请不要在正式论文撰写中如本文般滥用脚注}}

% 原则上,每一章开头不可马上进入第一节
% 需要一段引文提要起到 TL;DR 的作用
使用 Unix 系操作系统尤其是 OS X 的同学越来越多,
而国内学校发布公共文档仍旧多使用微软的Office系列和windows操作系统的内置字体。
在这种矛盾下,
惯用Unix系的同学在通过相应兼容方案打开或编辑此类文档时遭遇诸多不便,
甚至被迫切换到虚拟机环境下撰写论文。
鉴于此,本模板解决的首要问题即是:
\textbf{使用OS X的同学如何完全脱离windows环境完成论文撰写工作}。

对于未参与长期科研工作且惯用 windows 操作系统的国内大学生而言,
提到“排版”,仅能考虑到的工具几乎只有微软公司的字处理软件Word系列。
然而,讽刺的是,认真掌握使用Word系列工具的同学并不占多数。
模板作者曾参与软件学院论文格式审查工作,
发现半数以上的同学连字体都无法按照学院要求设定,
相当一部分同学目录、图片、参考文献排版混乱不堪。
简单分析其中原因,概有以下几条:
\begin{enumerate}
    \item 对论文撰写工作不够重视,将学位论文当成项目报告
    \item office系列软件的“易用性”导致了在工具使用上的心理轻视
    \item 缺乏对排版美观的感受能力
\end{enumerate}

工具本身并无高低贵贱之分,
无论使用何种工具,都有方法高效地排版一份精美的论文。
使用\LaTeX \footnote{知道准确的\LaTeX 发音的不必嘲笑将它念成leiteks的人}
并不是以此体现自己高人一等的装x行为\footnote{此处不讨论鄙视链和党争等哲学话题},
参看王垠的《谈Linux,Windows和Mac》\cite{yinwang2013}。
它只是另一种选择而已。
习惯在windows下工作但暂时不能熟练使用 word 系列排版论文的同学,
也欢迎使用 \LaTeX 完成您的论文排版,祝您打开一个新世界。

\section{\LaTeX  简介}
一本图书出版的第一步一般是由作者将他们的手稿交给出版公司,
由专业的排版人员负责整本书的排版,
而学位论文的撰写工作其实同时包括了这两个程序。
本模板发布的初衷即是把同学们在排版方面的时间成本降到最小,
得以保证足够的精力\textbf{专注于内容本身}的撰写。

\subsection{\LaTeX 极简史}
四十年前,高德纳为了亲自排版他的《计算机程序设计艺术》,
编写了\TeX 排版引擎。
核心部分是一个名为\texttt{tex}的程序,
这个程序将使用 TeX Primitive 格式编写的排版指令编译成用于打印的文件。
而本文所提\LaTeX 是一个建立在TeX Primitive 之上的宏包,
每一个\LaTeX 命令会被 \texttt{latex} 程序解释成几个甚至几百个\TeX 命令。

为了不增大同学们的学习成本,
关于 TeX, pdfTeX,  XeTeX,  LuaTeX 等复杂的历史脉络本文不再展开。
其实以上提到四种都是排版引擎,
而通过\LaTeX 格式编写的代码在以上四种引擎中都有对应的编译工具,
即 latex, pdflatex, xelatex, lualatex 。

原始的latex 编译系统不支持东方文字(CJK字符)
\footnote{包括大陆新加坡简体汉字、港台繁體漢字、日本戦後新字体、戰前舊字體、カタカナ、ひらがな、언문等}
排版。
本模板默认使用使用 xelatex 引入 XeCJK 宏包解决中文和西文混排问题。

\subsection{发行版和文本编辑器}
\begin{verbatim}
地址: https://github.com/KwenString/Thesis-SE-ZJU-LaTeX
\end{verbatim}
如果你能通过README编译出此份文档,
说明您的机器上已经安装好了一个\TeX 排版系统的发行版(比如MacTeX 2015)。
发行版融合了非常多的工具(包括命令行工具和窗口程序)和宏包,
像Linux的发行版一样有bash, coreutil, gcc 这些工具集。
您在本项目README中看到的\texttt{latexmk}就是发行版包含的一个重要命令行工具。
和Linux发行版一样,由于排版系统的发行版也需要包管理,
在 OS X 下,可以使用 MacTeX 自带的图形化包管理工具 TeX Live Utility,
也可以使用命令行工具\texttt{tlmgr}操作。

论文的\LaTeX 源码自然是纯文本,
您可以使用平时编码时最惯用的文本编辑器完成。
若您仅通过集成开发环境(IDE)完成编码工作,
那推荐使用简单易学的Sublime Text
\footnote{请不要陷入无意义的编辑器党争,适合自己的才是坠吼的。}
安装相应语法高亮插件。
使用windows的同学,切记不可使用在纯文本前擅自插入BOM标记的
windows记事本程序,另外,文本保存时确保为utf-8格式。

“纯文本”是使用\LaTeX 排版的第二个优势,
纯文本文件稳定易读,可以很方便地通过各种源码版本控制工具进行版本管理,
但请注意保护自己的论文源码,避免无意上传到线上公开仓库。

本模板使用\LaTeX 格式编写论文源码。
\label{dirtree}
接下来介绍一些\LaTeX 必要的编写语法和基本知识,
以便同学们在遇到问题时能尽量准确的描述以寻求社区或个人协助。
现在您可以分屏一边阅读本文一边阅读本文源码
(本章源码在\texttt{contents/whyla.tex})
来最直观地了解基本语法。

% TikZ 大法 先跳过
\begin{figure}[htbp]
    \centering
    \begin{tikzpicture}[dirtree]
      \node {论文根目录}
        child { node {contents/}
            child{ node {abstract\_chinese.tex}} 
            child{ node {abstract\_english.tex}} 
            child{ node {elem.tex}}
            child{ node {intro.tex}}
            child{ node {whyla.tex}}
            child{ node {rule.tex}}
            child{ node {end.tex}}
            child{ node {thanks.tex}}
        }		
        child { node {main.tex}}
        child { node {figures/}}
        child { node {references/}}
        child { node {gbt7714-2005.bst}}
        child { node {zjuthesis.cls}};
    \end{tikzpicture}
    \caption{论文源码树}
\end{figure}
% 如果您在这里忘记逃逸下划线,靠编译错误信息根本无法发现
% 如果您不知道我在这里说什么,请继续往后看

在文本编辑器中另开一个标签或分区打开\texttt{main.tex},
从\texttt{\textbackslash documentclass\{...\}}
到\texttt{\textbackslash begin\{document\}}
之间的部分称为导言区。
此部分通常用于全文的样式设定。
为尽量分离样式和内容,
本模板将样式文件独立至\texttt{zjuthesis.cls}
(请优先选用只读模式打开样式文件),
导言区仅描述了插图引用的目录和样式表引用的变量值。
在导言区之后,即是文档内容的撰写,
鉴于论文篇幅较大,本模板建议按章节分立论文源码。

\section{探索学习}
如本节标题,本文不会系统地介绍\LaTeX 的各项语法标记。
若您觉得确有必要系统地学习\LaTeX 语法知识,
请使用搜索引擎搜索关键字:一份不太简短的 LaTeX 2ε 介绍。

通过调查本章的源码文件,
相信你已经懂得了如何开始编写
一章(\texttt{chapter})、
一节(\texttt{section})
和一小节(\texttt{subsection})。
% 如果需要编写更小的文档结构,
% 可以使用(\texttt{subsub-section})
% \footnote{此处标记实为\texttt{subsubsection}。
% 在连续排版等宽字体时,
% 断行算法容易失控造成溢出版心,
% 如果您认为应极力避免一切溢出版心的排版行为时,
% 请仅在文档内容稳定后再对此细节做调整。}。
强烈不建议使用到第四级标题x.x.x.x,
学院论文模板要求从第四级标题开始不得进入目录。
考虑到这种需求多见于论文第二章所谓相关技术介绍时,
部分同学列举一些背景概念时使用,
从美观角度和大部分学位论文的篇幅考虑,
本模板建议使用子段落(\texttt{subparagraph})来实现这个语义,如下。

% htbp 什么的现在不要管
\begin{figure}[htbp]
    \centering  % 学位论文规定图表皆水平居中**版心** 在 zjuthesis.cls 搜「版心设置」
    \includegraphics[width = .4\linewidth]{plus-1.jpeg} % 设定图片宽度相对于**版心**宽度,图片文件资源名
    \caption{华莱士在其著作《马来群岛》中绘制的飞蛙速写} % 图的题注
    \label{fig:plus-1} % 与 autoref 关联,设定交叉引用和显示「图x.x」
\end{figure}


% 梦里不觉__已深, ____岂是为他人。
\subparagraph{华莱士飞蛙}
华莱士飞蛙 (Rhacophorus nigropalmatus) ,
得名于它的发现者——生物学家阿尔弗雷德 · 华莱士 (Alfred R. Wallace)。
华莱士飞蛙生活在马来半岛的森林里,
它的体型很大,体长有 8 到 10 厘米,
除了交配和产卵,它们几乎从不下树……
如\autoref{fig:plus-1} 所示。

如你从本章源码所见,单个换行符并不会编译成一个换行符,
而两个或者超过两个换行符将被解释成一个分段,
类似HTML中新建了一个\texttt{<p>}标签。
如果需要在文章中随意插入一个换行,
则需要在源码文件中编写$\backslash\backslash$实现。
注意此标记一般仅在排版表格时,
或者后期调整少数溢出版心的情况时使用。

由于\LaTeX 的命令会使用几个固定的字符,
同各种编程语言处理字符串时一样,
当输出此类字符时需要使用一定的逃逸策略,
如果您的论文编译错误,请\textbf{优先检查是否在源码输入了未经逃逸的字符}
\footnote{就连写这份文档的时候都还是不记得给下划线逃逸}。
逃逸规则见\autoref{tab:escape} 所示。


% 和图一样,这里的htbp先不要管,先照抄
\begin{table}[htbp]
    \centering  % 依照规定 表格必须居中版心放置
    \caption{\LaTeX 命令专用字符逃逸规则} % 表格题注,zjuthesis.cls 将其设置在表格之上
    \label{tab:escape} % 交叉引用
    \begin{tabu}{lllllllll} % 9 列均左对齐
        \toprule % 头线
        输出 & \#  & \$  & \&  & \_  & \{   & \}  & \~{}  & \`{} \\ % 注意到了这个换行符吗
        \midrule % 中线
        源码 &  % 感受一下这里的「逃逸」
        \texttt{$\backslash$\#} &
        \texttt{$\backslash$\$} &
        \texttt{$\backslash$\&} &
        \texttt{$\backslash$\_} &
        \texttt{$\backslash$\{} &
        \texttt{$\backslash$\}} & 
        \texttt{$\backslash$\~{}} &
        \texttt{$\backslash$\`{}}\\
        \bottomrule % 底线
    \end{tabu}
\end{table}

至于反斜杠本身的逃逸方法,请见上一段文字或表格的源码。

默认情况下,一行文字的源码里,
多个空格符、水平制表符或单个换行符都仅会被编译成一个空格,
然而,在\textbf{中文环境}下,并不需要通过空格实现如西文那样的分词,
所以本模板在XeCJK环境的初始化配置上取消了如此产生空格的规则。
利用这样的改动,
您在编写论文源码时可随时换行,
不必在意会产生多余的空格。
随着您的行文篇幅渐巨,
您将慢慢体会到何为\textbf{内容}与\textbf{样式}分离的思想。

关于输出英文半角双引号,需要在源码编写两个反引号(``double grave'')。
而中文双引号则直接在源码中编写一对全角双引号即可。
鉴于目前大陆地区的官方标准明确规定使用此种方法标记引号,
请不要在论文中使用台湾地区和日本等地使用的直排引号
\footnote{网页中文排版习惯近年来有惯用直排引号的趋势,不做讨论}。

% 论文除了第一章绪论和终章总结与展望, 最后都需要撰写一节“本章小结”。
% 活用注释作为 ToDo 提醒是一个好习惯
\section{本章小结}
这一章简单描述了\LaTeX 排版的基本概念和本模板的源码结构,
通过同步实例介绍了论文模板最基本语义单元的编写方法。
现在您可以尝试去编写自己的论文内容,
当遇到无法排版的元素或不可调和的矛盾时,可以继续阅读下一章的内容。


%!TEX root = ../main.tex

\chapter{学位论文排版元素}

通过阅读上一章节,
相信您已基本配置好了自己的编辑环境,
了解了如何正确编写各级章节、段落和子段落,
容易引起编译出错的逃逸字符。

从上一章出现过的有序列表、外部图片和代码环境中,
您应该已经发现,
使用命令\texttt{$\backslash$begin\{xxxx\}}可以开始一段新的布局环境。
这一章将系统地阐述计算机类的学位论文需要排版的元素的编写方法。
主要包括以下排版要素:
\begin{itemize}
    \item 列表环境(包括有序、无序、定义三种列表)
    \item 插图和表格
    \item 代码环境
    \item 数学和算法环境
\end{itemize}

这一章只涉及为完成此排版环境的宏包的最基本使用方法,
本章尽量覆盖论文写作中的大部分场景,
如有特殊需求,请仔细阅读相关宏包手册并求助于国内外TeX社区及问答网站。

\section{列表环境}

列表环境有三种,
类似与HTML的\texttt{<ol>},
\texttt{<ul>},\texttt{<dl>}
三个标签。
以下是一个定义列表环境:
\begin{description}
    \item[有序列表] enumerate 默认从阿拉伯数字1开始编号,如需更改请搜“重定义列表”
    \item[无序列表] itemize 默认圆点标记,尽量少用此类列表
    \item[定义列表] description 语义上用于作一系列简短的解释
\end{description}

列表可以嵌套,比如:
\begin{enumerate}
	\item 第一级列表
	\item 第一级列表
	\begin{enumerate}
		\item 第二级列表
		\item 第二级列表
        \begin{itemize}
            \item 第三级列表
            \item 第三级列表
		\end{itemize}
		\item 第二级列表
		\item 第二级列表
	\end{enumerate}
\end{enumerate}

\section{插图环境和浮动体}

相信您在上一章的探索学习中已经基本掌握了如何插入图片的方法,
但可能仍存疑虑。
所以现在先简单介绍浮动体的概念以助您理解插图环境的布局规则,
最后再介绍子图的排布以应对您更高的布局需求。
% 关于绘图,本文将在后续章节讲述
% 活用引用,让评阅老师随处移动
关于图的绘制,本文将在\ref{how-to-plot} 继续讲述。

当一个图片或表格太大在当前页面无法继续排布时,
一种解决方案就是新开一页再排布(Word 默认使用此种)。
这个方法在页面上留下分空白,十分不美观。
\LaTeX 的默认解决方案是把它们“浮动”到下一页,
与此同时使用后续正文文本填充当前的空白。

插图和表格在\LaTeX 排版中被默认当成浮动体对待,
当排版引擎试图放置一个浮动体时,它将遵循以下规则:
\begin{enumerate}
    \item 浮动体的布局大小不得超过版心的大小,否则抛出Overfull Page错误
    \item 浮动体只会向后浮动,不会向前浮动
    \item 浮动体默认按照 h $\to$ t $\to$ b $\to$ p 的规则布局
    \begin{description}
        \item[h] 当前位置,如果本页所剩空间不够,这一参数无效
        \item[t] 浮动到下一页顶部
        \item[b] 浮动到下一页底部(脚注之下)
        \item[p] 浮动到一个允许出现浮动体的页面上
        \item[!] 忽略浮动体放置的大多数内部参数\footnote{作者也不太懂}
    \end{description}
    \item 设置 htbp 参数的顺序不会影响默认的规则顺序
\end{enumerate}

在实践中,一般选用浮动规则[htbp], [tbp], [htp], [tp] 来完成浮动体布局。
请不要使用单一参数布局,这样极有可能出现难解的浮动问题。
不适当的浮动规则参数将导致浮动对象被放进一个队列中等待布局,
这个队列的默认大小是18,如果队列超限,编译中会抛出一个Too Many Unprocessed Floats错误。
如果一页图片太多,甚至几乎占满一个版心,
您可以通过\texttt{$\backslash$clearpage}命令强制在此处必须排版完所有浮动体
再排版之后内容,关于清除浮动等复杂主题,这里不再展开。
此处只能建议插图尺寸不宜过大,插图密度不宜过大。
考虑到图文混排的最佳视觉要求,
可以待论文内容稳定下之后,仔细调整插图代码的位置,
通过前置插图代码,强行“向后浮动”,保证插图和引用处足够近。

关于模板对浮动体的设置,参看\texttt{zjuthesis.cls},
搜索关键字“浮动体”找到对应配置。
图片引用路径在\texttt{zjuthesis.cls}里定义的\texttt{graphicspath}里,
默认情况下\\\texttt{$\backslash$includegraphics}命令从论文源码根目录搜索引用的图片,
如果先在根目录里匹配到文件名,则不再前往定义路径搜索,
当引擎无法找到您指定的图片资源时,会导致编译错误。
注意引用的文件名包括文件后缀。

% 现在你可以随意更动此插图代码的位置来感受一下浮动体布局的规则
\begin{figure}[htbp]
	\centering
	\begin{subfigure}[b]{.45\textwidth}  % 注意此处的尺寸控制
		\centering
		\includegraphics[width = \textwidth]{xuejian.jpg}
		\caption{仙三}\label{fig:subfig-samp1}
	\end{subfigure}
	\begin{subfigure}[b]{.45\textwidth}
		\centering
		\includegraphics[width = \textwidth]{wenhui.jpg}
		\caption{仙三外}\label{fig:subfig-samp2}
	\end{subfigure}
	\begin{subfigure}[b]{.45\textwidth}
		\centering
		\includegraphics[width = \textwidth]{lingsha.jpg}
		\caption{仙四}\label{fig:subfig-samp3}
	\end{subfigure}
	\caption{仙剑白学传}\label{fig:subfig-samp}
\end{figure}

接下来描述一下子图的编写,
在实际论文撰写过程中,
经常遇到需要比较几组实验数据或场景的需求。
此时,合乎语义的做法是为不同的组设置子图,
而不是分别设图。

多个子图组成一个单独的浮动体进行布局,
共用一个总图题总引用,并可以有各自单独的子图题和交叉引用。
本模板使用subcaption 宏包处理子图排版问题,如\autoref{fig:subfig-samp} 所示
\footnote{不要在正式论文排版过程中使用彩色区分类别,论文最终以灰度打印}。
论文中不可像本文一般,
平白无故地出现与行文毫无关联的图例,
而且,必须有适当的文字内容对图例做出解释。
比如比较分析从\autoref{fig:subfig-samp1} 到\autoref{fig:subfig-samp3}
仙剑系列在白学梗方面的运用变迁。\footnote{往后数代仍有类似场景 -\_-\# (顔文字書込禁止!)}

当准备插图资源时,应该尽可能保证插图清晰,背景透明。
图中文字大小与文中接近,不小于脚注大小,不大于正文段落文字大小,
框线宽度不大于2px。

如果您曾关注过图片格式,
应该知道图片在计算机中一般分为矢量图(\autoref{fig:vector})和位图(\autoref{fig:raster})两种类型。
通俗地理解,矢量图通过几何属性存储信息,所以在缩放时保持图形的几何属性。
而位图按像素点存储信息,在缩放时必然丢失信息。
对于学位论文里的大部分为表达实验数据而描述的图例,最好使用矢量图绘制,
以给评阅老师或后人精确地参考和还原实验。
常用的矢量图格式有eps, pdf, svg 和 Adobe 系列的文件格式。
其中\LaTeX 格式可以直接引用eps 和 pdf 格式的图片。

\begin{figure}[htbp]
	\centering
	\begin{subfigure}[b]{.45\textwidth}
		\centering
		\includegraphics[width = \textwidth]{vector.pdf}
		\caption{矢量图}\label{fig:vector}
	\end{subfigure}
	\begin{subfigure}[b]{.45\textwidth}
		\centering
		\includegraphics[width = \textwidth]{raster.png}
		\caption{位图}\label{fig:raster}
	\end{subfigure}
	\caption{Google Logo 的矢量图和位图比较}\label{fig:vector-raster}
\end{figure}


\section{表格}
表格与插图一样,也是浮动体。
在\LaTeX 中,表格的编写成本比较高,
极易引发编译错误。
对于只有两列的表格,建议改用列表环境完成排版。
本模板使用tabu环境排版表格,
使用longtabu环境排版超长表格。
学术论文多用线条简洁的三线表,
所谓三线就是 toprule, midrule和bottomrule 。
如\autoref{tab:tabu_test_1} 是对tabu宏包的tabu表格环境测试。
\begin{table}[htbp]
	\centering
	\caption{这是一个用tabu环境的测试用的表格}\label{tab:tabu_test_1}
    \begin{tabu}{lrr} % lrr 表示 左对齐 右对齐 右对齐
    %\begin{tabu}{|l|r|r|} % 加上竖线看看

    \toprule % 软件学院论文模板规定表头必须加粗
    \textbf{行星}     & \textbf{赤道半径}km & \textbf{公转周期}d \\
    \midrule
    水星     & 2.439  & 87.9 \\
    金星     & 6.1    & 224.682 \\
    地球     & 6378.14 & 365.24 \\
    \bottomrule
    \end{tabu}%
\end{table}

\autoref{tab:tabu_test_2} 对tabu to表格的x列模式进行测试。在表格导言区中设置为X[1]X[2]X[2],表示这三列表格的列宽比值为1:2:2,总的表格宽度由tabu to环境设置,这里设置为0.6\textbackslash linewidth。相比于tabular环境,tabu环境的列宽设置方便许多。
\begin{table}[htbp]
	\centering
	\caption{tabu环境测试表格---X列模式}\label{tab:tabu_test_2}
    \begin{tabu} to 0.6\linewidth{X[1]X[2]X[2]}
    \toprule
    \textbf{行星}     & \textbf{赤道半径}km & \textbf{公转周期}d \\  % 为了表格排版的美观 表头建议加粗
    \midrule
    水星     & 2.439  & 87.9 \\
    金星     & 6.1    & 224.682 \\
    地球     & 6378.14 & 365.24 \\
    \bottomrule
    \end{tabu}%
\end{table}

如\autoref{tab:tabu_test_3}是longtabu环境测试表格。
longtabu环境不能用在table浮动体环境中。
根据GB/T 7713.1-2006规定:如果某个表需要转页接排,
在随后的各页上应重复表的编号。
编号后跟标题(可省略)和“(续)”, % 表:「我要续…… +1
置于表上方。
续表应重复表头。

特别需要注意的是,
longtabu是基于longtable宏包开发的,
所以在zjuthesis.cls文件中已经插入了longtable宏包。
longtable环境的所有功能都可以在longtabu中使用,
如\textbackslash endhead,
\textbackslash endfirsthead,
\textbackslash endfoot,
\textbackslash endlastfoot,
和\textbackslash caption等。
具体用法请参见longtable和tabu宏包的相应文档。

\begin{longtabu}{lccc}
\caption{材料弹性模量及泊松比}\label{tab:tabu_test_3}\\
\toprule
名  称   & 弹性模量E/Gpa & 切变模量G/Gpa & 泊松比$\mu$ \\
\midrule%
\endfirsthead
\caption{材料弹性模量及泊松比(续)}\\
\toprule
名  称   & 弹性模量E/Gpa & 切变模量G/Gpa & 泊松比$\mu$ \\
\midrule%
\endhead
\bottomrule%
\endfoot
镍铬钢、合金钢 & 206    & 79.38  & 0.3 \\
碳 钢    &  196~206 & 79     & 0.3 \\
铸 钢    &  172~202 &        & 0.3 \\
球墨铸铁   &  140~154 &  73~76 & 0.3 \\
灰铸铁、白口铸铁 &  113~157 & 44     &  0.23~0.27 \\
冷拔纯铜   & 127    & 48     &   \\
轧制磷青铜  & 113    & 41     &  0.32~0.35 \\
轧制纯铜   & 108    & 39     &  0.31~0.34 \\
轧制锰青铜  & 108    & 39     & 0.35 \\
铸铝青铜   & 103    & 41     & 0.3 \\
冷拔黄铜   &  89~97 &  34~36 &  0.32~0.42 \\
轧制锌    & 82     & 31     & 0.27 \\
硬铝合金   & 70     & 26     & 0.3 \\
轧制铝    & 68     &  25~26 &  0.32~0.36 \\
铅      & 17     & 7      & 0.42 \\
玻璃     & 55     & 22     & 0.25 \\
混凝土    &  14~39 &  439~15.7 &  0.1~0.18 \\
纵纹木材   &  9.8~12 & 0.5    &   \\
横纹木材   &  0.5~0.98 &  0.44~0.64 &   \\
橡胶     & 0.00784 &        & 0.47 \\
电木     &  1.96~2.94 &  0.69~2.06 &  0.35~0.38 \\
赛璐珞    &  1.71~1.89 &  0.69~0.98 & 0.4 \\
可锻铸铁   & 152    &        &  \\
拔制铝线   & 69     &        &  \\
大理石    & 55     &        &  \\
花岗石    & 48     &        &  \\
石灰石    & 41     &        &  \\
尼龙1010 & 1.07   &        &  \\
夹布酚醛塑料 &  4~8.8 &        &  \\
石棉酚醛塑料 & 1.3    &        &  \\
高压聚乙烯  &  0.15~0.25 &        &  \\
低压聚乙烯  &  0.49~0.78 &        &  \\
聚丙烯    &  1.32~1.42 &        &  \\
硬聚氯乙烯  &  3.14~3.92 &        &  \\
聚四氟乙烯  &  1.14~1.42 &        &  \\
\end{longtabu}%


\section{代码段}

原则上,论文中应尽量少的出现代码段。
如果不得不引用一小部分代码,
可以使用\texttt{lstlisting}设置代码环境。
本模板的代码环境默认配置在\texttt{zjuthesis.cls}搜索关键字“代码”。

因本模板不鼓励引用大段代码,
所以默认情况下不为代码开启行号。
观查\autoref{code:samp-code},结合前述图表设置,
试图理解代码环境的编写。

\begin{lstlisting}[language=C++,numbers=left, numberstyle=\tiny,label=code:samp-code, caption=一段Chromium的源代码]
// Start tasks to take all the threads and block them.
  const int kNumBlockTasks = static_cast<int>(kNumWorkerThreads);
  for (int i = 0; i < kNumBlockTasks; ++i) {
    EXPECT_TRUE(pool()->PostWorkerTask(
        FROM_HERE,
        base::Bind(&TestTracker::BlockTask, tracker(), i, &blocker)));
  }
  tracker()->WaitUntilTasksBlocked(kNumWorkerThreads);

  // Setup to open the floodgates from within Shutdown().
  SetWillWaitForShutdownCallback(
      base::Bind(&TestTracker::PostBlockingTaskThenUnblockThreads,
                 scoped_refptr<TestTracker>(tracker()), pool(), &blocker,
                 kNumWorkerThreads));
  pool()->Shutdown(kNumWorkerThreads + 1);

  // Ensure that the correct number of tasks actually got run.
  tracker()->WaitUntilTasksComplete(static_cast<size_t>(kNumWorkerThreads + 1));
  tracker()->ClearCompleteSequence();
\end{lstlisting}

引用一两行代码,可以直接使用\texttt{verbatim}环境完成。
注意此环境不会采取任何主动断行策略。
\begin{verbatim}
Error: Command failed: /bin/sh -c rsync -arvq --exclude cache
--exclude .git 
\end{verbatim}

\section{数学和算法环境}

\TeX 模板引擎创立之初就是为了最美观地排版本节的主题。
在理工科的学位论文中,数学符号和数学公式必不可少
\footnote{至于定理、引理和推论等纯理科环境,本模板未作任何设定,不讨论。}。
在本模板中,数学环境由amsmath和amssymb宏包支持。
即便没有使用公式,您应该也希望看到$a_1$, $a_2$, $a_3$而不是a1, a2, a3 吧?

于简单的行内公式,
直接在源码处编写\texttt{\$...\$}内的公式即可,
不熟习\LaTeX 公式编写的同学,
可以使用可视化的公式编辑器产生\LaTeX 代码,
这里推荐使用Daum Equation Editor完成复杂公式编辑的工作。

对于单行公式,可以使用\texttt{\$\$...\$\$}创建。
$$Y=\sum_{k=1}^n X_k$$
如果需要设定交叉引用,那推荐align环境创建,如\eqref{eq:samp}所示。
\begin{align}\label{eq:samp}
    f(x) & = 2(x + 1)^{2} - 1\\                  % & 用来对齐等号
		 & = 2(x^{2} + 2x +1)-1\\
		 & = 2x^{2} + 4x + 1
\end{align}

%一个矩阵
%$$\begin{bmatrix}
%1&2&3&4\\
%5&6&7&8\\
%9&10&11&12
%\end{bmatrix}$$

计算机类的学位论文
一般少不了对研究算法的描述。
本模板选用algorithmi2e宏包排版算法环境。
详细指令使用方式参见宏包使用手册
\footnote{一般有需求排布复杂算法的同学应该有一定的科研经历}。
如\autoref{algo:duplicate2}

\begin{algorithm}
\DontPrintSemicolon
\KwIn{A sequence of integers $\langle a_1, a_2, \ldots, a_n \rangle$}
\KwOut{The index of first location with the same value as in a previous location in the sequence}
$location \gets 0$\;
$i \gets 2$\;
\While{$i \leq n \land location = 0$} {
  $j \gets 1$\;
  \While{$j < i \land location = 0$} {
    % The "l" before the If makes it so it does not expand to a second line
    \lIf{$a_i = a_j$} {
      $location \gets i$\;
    }
    \lElse{
      $j \gets j + 1$\;
    }
  }
  $i \gets i + 1$\;
}
\Return{location}\;
\caption{{\sc FindDuplicate2}}
\label{algo:duplicate2}
\end{algorithm}

\section{绘图}\label{how-to-plot}

一图胜千言,经过同学们辛苦的实验积累下的数据,
应该尽量以最高质量呈现出来,而不是使用冗长的语句反复言说。
使用强大的TikZ宏包,可以绘制各式各样的图例,
比如在\ref{dirtree} 的目录结构图就是使用TikZ宏包绘制完成。
通过绘图宏包得到的是矢量图,
经过缩放后仍能精确地指导打印。
由于使用TikZ宏包绘制各式图例的方法艰深繁杂,
非长期钻研学术者实不可速取。

% 这里删掉了一大段TikZ宏包的使用
% 太复杂了  如果不是跟老师搞学术的话真的算了

含有大量数据的统计图,
从事数据分析工作的同学可自行使用python或R语言完成绘制,
确保输出eps或pdf格式图形,使用插图环境引入即可。

对于一般的流程图,本模板推荐使用graphviz绘图工具绘制。
相对于TikZ,graphviz已经足够适合人类掌握了。
如果坚持使用可视化工具完成此类图例的绘制,
本文推荐一个在线绘图工具\texttt{https://www.draw.io},
该工具可以绘制流程图、UML系列和很多可以引入的小图标。
另外它还支持dropbox同步及输出pdf,通过同步论文的图片引用目录,
可以最高效的完成绘图和插图的工作。

无论用何种工具完成绘图,
时间精力成本都不会太低。
请妥善规划您的论文撰写时间,
确保顺利毕业。

\section{关于参考文献}

硕士学位论文的参考文献学院规定至少20篇以上,
不可滥引,注重引文质量。

参考文献标准参照国家标准《GB/T 7714-2005: 文后参考文献著录规则》\footnote{此标准规定的学位论文引用格式并无指定需列出是“硕士学位论文”还是“博士学位论文”}。
样式文件由南京大学胡海星提供。
\begin{verbatim}
http://haixing-hu.github.io/nju-thesis/
\end{verbatim}

参考文献采用顺序编码制,即引文处采用序号标注,参考文献表按引文序号列出。
参考文献的排版需要引入同学们自己的参考文献数据库,
南京大学胡海星提供了一个样例数据库,见\texttt{references/test.bib}。
建议通过各式文献管理工具,在论文早期工作时逐渐积累文献数据库。
通过Google学术查找一篇文献时,如\autoref{fig:gscholar} 所示,点击cite,
选择BibTeX,即可得到本文献的Bib格式的各项字段。
\begin{figure}[htbp]
    \centering
    \includegraphics[width=\textwidth]{gscholar.png}
    \caption{使用Google学术查找引文的BibTeX字段}
    \label{fig:gscholar}
\end{figure}

注意,由Google学术提供的文献类型和字段有可能不满足胡海星的设定,
注意调整。
胡海星提供的参考文献样式表中设定的文献类型列出:
\begin{description}
    \item[期刊]          \texttt{@article}
    \item[专著]          \texttt{@book, @inbook}
    \item[译著]          \texttt{@Book, @inBook}
    \item[会议论文集]    \texttt{@proceeding, @inproceeding}
    \item[手册]          \texttt{@manual}
    \item[网页]          \texttt{@webpage, @online}
\end{description}

\begin{itemize}
    \item 比如这是一篇中文期刊\cite{lixiaodong1999}
    \item 这是几篇英文期刊\cite{christine1998, kanamori1998}
    \item 一本中文书\cite{zh-book-1}
    \item 一本中文译著\cite{anwen1988b}
    \item 一本英文书\cite{lamport1994latex, takeuti1973}
    \item 一篇中文inproceeding\cite{nonlinear1996}
    \item 中文proceeding\cite{a2-1}
    \item 英文proceeding\cite{a2-2}
    \item 中文inproceeding\cite{aczel1998}
    \item 一篇学位论文\cite{a4-1} 
    \item 其他资料:手册\cite{ipad}报纸\cite{renminribao}网页\cite{dubash2010}
\end{itemize}

在论文中设置了一个错误或丢失的引用不会引起编译错误,
引擎会在引用标记中设一个问号。
手动编译论文的顺序一般为:
\begin{verbatim}
xelatex main
bibtex main  // 生成参考文献
xelatex main
xelatex main
\end{verbatim}
latexmk 自动化地执行了这些步骤,所以编译时间才需要20余秒之久。

\section{本章小结}

本章划分节比较多,正式行文中请尽量避免。

传播智识,单单借助文字的力量是相对无力的,
即使是日常博文,列表、插图、表格、代码都少不了。
何况是一篇用于申请硕士学位的论文呢?

一篇学位论文集长期的科研工程实践智慧于寥寥数万字。
如何合理规划论文语义和排版元素,
让即便不熟习此领域的后人能在短时间内消化,
继续开物前民,
是同学们应该在论文撰写过程中反复求索的。



%!TEX root = ../main.tex

\chapter{总结和展望}

\section{相关工作总结}

本模板主要内容来源于ZJUawesome项目,
参考软件学院论文格式要求做出调整,
并加入补充宏包,调整若干属性配置完成。
封面方面主要调整了各栏间距和对齐,
摘要依照软件学院更改了关键字的样式和页码的样式。
软件学院规定从摘要起每页必须有对应章标题的页眉,
虽然在章头处排版页眉本不雅观,
但考虑到已经有以大部分同学使用Word字处理软件遵照执行,
为保一致性,本模板暂时向软件学院的设定妥协。
由于论文格式要求并未向章头处的间距做出任何设定,
本模板保留ZJUawesome设定。
除此之外,本模板还做出了不少微小的改动。
详情请仔细阅读\texttt{zjuthesis.cls}和\texttt{main.tex}相关内容。

考虑到大部分软件学院的同学对\LaTeX 论文排版的陌生,
本文以尽量精炼的篇幅介绍了论文排版工作的各方面。
现在给出一个参考流程如\autoref{fig:workflow} 希望能对初次使用
\LaTeX 排版论文的同学一点提示。

\begin{figure}[htbp]
    \centering
    \includegraphics[width=.8\textwidth]{workflow.pdf}
    \caption{论文排版工作参考流程}
    \label{fig:workflow}
\end{figure}

\section{展望}

本文没有讨论各式\LaTeX 环境的使用细则,宏包的具体细节,
调试查错的技巧,也几乎没有交待任何一种具体的绘图方案。
本文希望屏幕前的你能以最少的时间代价完成论文排版工作,
养成到内容和样式完全分离的电子写作习惯,
并在对外输出知识时常思考最佳信息表达的模式。

本文草拟于2016年夏季毕业论文送审前,
希望本文能抛砖引玉。
对\LaTeX 有经验的后辈们若能继续完善甚至颠覆本模板的设定,
相信一定能对软件学院的论文排版素质起到根本的改善作用。

\section{真实参考资料}

考虑到下一页的参考文献是样例凑数,
此处不完整且不严谨地列出本文真实的参考资料:
\begin{description}
    \item[常规文档] \texttt{http://latexfly.com/docs/}
    \item[算法宏包] \texttt{https://en.wikibooks.org/wiki/LaTeX/Algorithms}
    \item[优雅的使用Word] \texttt{https://www.zhihu.com/question/20541531}
    \item[电子科大论文模板] \texttt{https://github.com/shifujun/UESTCthesis/wiki}
    \item[实时编译] \texttt{http://xiaoweiz.github.io/posts/2014/Aug/ST\_skim\_latexmk/}
    \item[一份不太简短的\LaTeX2e 介绍]   
    \item[各种宏包的手册] 在 TeX Live Utility 就可以查看
\end{description}
  % 总结和展望

% 结尾部分排版
\backmatter

% 引用参考文献数据库
\bibliography{references/test.bib}

% 附录部分
%\appendix
%% !TEX root = ../main.tex
\chapter{我是第一个附录}
\section{我是第一个附录的第一节}
这是一个附录测试页,内容无关紧要。\footnote{以下内容引用自《三体:黑暗森林》}以%
下段落较长,以防数组溢出,故采用回车强制分行处理。分行出换行符在\TeX 中算作一个%
空格,因此,在每段后加注释符。不过在中文环境中换行加不加注释符都不会产生空格,不%
过还是加上吧。

罗辑抬起左手,露出了戴在手腕上的手表大小的东西说:“这是一个生命体征监测仪,它通%
过一个发射器与一套摇篮系统联结。你们一定记得两个世纪前面壁者雷迪亚兹的事,那就一%
定知道摇篮系统是什么。这个监测仪所发出的信号通过摇篮系统的链路,到达雪地工程部署%
在太阳轨道上的三千六百一十四枚核弹。

信号每秒钟发射一次,维持着这些核弹的非触发状态。如果我死去,摇篮系统的维持信号将%
消失,所有的核弹将被引爆,包裹核弹的油膜物质将在爆炸中形成围绕太阳的三千六百一十%
四团星际尘埃,从远方观察,在这些尘埃云团的遮挡下,太阳将在可见光和其他高频渡段发%
生闪烁。太阳轨道上所有核弹的位置都是经过精心布置的,使得太阳闪烁形成的信号发送出%
三张简单的图形,就像我两个世纪前发出的那三张图一样,每张上面有三十个点的排列,并%
标注其中一个点,它们可以组合成一个三维坐标图。但与那次不同的是,这次发送的,是三%
体世界与周围三十颗恒星的相对位置。太阳将变成银河系中的一座灯塔,把这咒语发送出去%
,当然,太阳系和地球的位置也会同时暴露。从银河系中的一点看,图形发射完成需要一年%
多的时间,但应该有很多技术发展到这样程度的文明,可以从多个方向同时观测太阳,那样%
的话,只需几天甚至几个小时,他们就能得到全部信息。”

\section{数学模式测试}
这里用于测试附录部分的数学公式,诸如标号,交叉应用等。

交叉引用测试,如交引用命令{\ttfamily \textbackslash eqref}和\texttt{\textbackslash ref}命令的区别。如公式\eqref{eq:apptest1},\autoref{eq:apptest1}显示,\texttt{\textbackslash eqref}命令比\texttt{\textbackslash ref}命令的应用结果多了个括号。

如公式\eqref{eq:apptest3}是单行公式环境,查看公式\eqref{eq:apptest3}和\eqref{eq:apptest1}之间的区别,好像在单行公式中没什么区别。
\begin{align}\label{eq:apptest3}
	f(x) = 2(x + 1)^{2} - 1
\end{align}

\texttt{align}公式环境,用在单行中。
\begin{align}\label{eq:apptest1}
	f(x) = 2(x + 1)^{2} - 1
\end{align}

在这里,中间插入一些文字以形成段落,查看行间公式与上下文之间的间隙。
\begin{align*}
	f(x) = 2(x + 1)^{2} - 1
\end{align*}
在这里,中间插入一些文字以形成段落,查看行间公式与上下文之间的间隙。下一个公式\eqref{eq:apptest2}是一个公式组,它在“=”位置对齐。
\begin{align}\label{eq:apptest2}
	f(x) & = 2(x + 1)^{2} - 1\\
		 & = 2(x^{2} + 2x +1)-1\\
		 & = 2x^{2} + 4x + 1
\end{align}

\subsection{我是第一个附录的第二节的第一个子节}

\section{表格测试}
在这里推荐制表采用功能强大的tabu宏包以取代其它制表宏包。具体tabu宏包的使用说明参见tabu宏包的说明文档。

以下节分别用来测试各种表格环境如,tabular,tabu,longtabu等,还有对caption格式的修改和测试。以下表格样式全部采用三线表。

\subsection{array宏包tabular表格环境测试}
如\autoref{tab:appfirst_table_test}是对array宏包的tabular表格环境测试。
\begin{table}[htbp]
	\centering
	\caption{这是一个用tabular环境的测试用的表格}\label{tab:appfirst_table_test}
    \begin{tabular}{lrr}
    \toprule
    \textbf{行星}     & \textbf{赤道半径}km & \textbf{公转周期}d \\
    \midrule
    水星     & 2.439  & 87.9 \\
    金星     & 6.1    & 224.682 \\
    地球     & 6378.14 & 365.24 \\
    \bottomrule
    \end{tabular}%
\end{table}

\subsection{tabu宏包表格环境测试}
如\autoref{tab:apptabu_test_1}是对tabu宏包的tabu表格环境测试。在这里表格命令与\autoref{tab:appfirst_table_test}的命令相同,只是tabular环境改成了tabu环境。
\begin{table}[htbp]
	\centering
	\caption{这是一个用tabu环境的测试用的表格}\label{tab:apptabu_test_1}
    \begin{tabu}{lrr}
    \toprule
    \textbf{行星}     & \textbf{赤道半径}km & \textbf{公转周期}d \\
    \midrule
    水星     & 2.439  & 87.9 \\
    金星     & 6.1    & 224.682 \\
    地球     & 6378.14 & 365.24 \\
    \bottomrule
    \end{tabu}%
\end{table}

\section{插图测试}
如\autoref{fig:appfirst_image_tset}是对此模版的第一张插图测试。

\begin{figure}[htbp]
	\centering
	\includegraphics[width = 0.5\linewidth]{Chapter8.png}
	\caption{附录页第一张插图测试}\label{fig:appfirst_image_tset}
\end{figure}

\section{我是第一个附录的第五节}
随着天光渐明,星星在一颗颗消失,仿佛无数只眼睛渐次闭上;而东方正在亮起的晨空,则%
像一只巨大的眼睛在慢慢睁开。蚂蚁继续在叶文洁的墓碑上攀爬着,穿行在她的名字构成的%
迷宫中。早在这个靠碑而立的豪赌者出现前的一亿年,它的种族已经生活在地球上,这个世%
界有它的一份,但对正在发生的事,它并不在意。

罗辑离开墓碑,站到他为自己挖掘的墓穴旁,将手枪顶到自己的心脏位置,说:“现在,我
将让自己的心脏停止跳动,与此同时我也将成为两个世界有史以来最大的罪犯。对于所犯下
的罪行,我对两个文明表示深深的歉意,但不会忏悔,因为这是唯一的选择。我知道智子就
在身边,但你们对人类的呼唤从不理睬,无言是最大的轻蔑,我们忍受这种轻蔑已经两个世
纪了,现在,如果你们愿意,可以继续保持沉默,我只给你们三十秒钟时间。”罗辑按照自
己的心跳来计时,由于现在心跳很急促。他把两次算一秒钟,在极度的紧张中他一开始就数
错了,只好从头数起,所以当智子出现时他并不能确定到底过了多少时间,客观时间大约流
逝了不到十秒钟,主观时间长得像一生。

这时他看到世界在眼前分成了四份,一份是周围的现实世界,另外三份是变形的映像。映像%
来自他前上方突然出现的三个球体,它们都有着全反射的镜面,就像他在最后一个梦中见到%
的墓碑那样。他不知道这是智子的几维展开,那三个球体都很大,在他的前方遮住了半个天%
空,挡住了正在亮起来的东方天际,在球体中映出的西方天空中他看到了几颗残星,球体下%
方映着变形的墓地和自己。罗辑最想知道的是为什么是三个,他首先想到的是三体世界的象%
征,就像叶文洁在最后一次ETO的聚会上看到的那个艺术品:但看到球体上所映照的虽然变%
形但异常清晰的现实图像时,他又感觉那是三个平行世界的入口,暗示着三种可能的选择;


% 作者简历
% !TEX root = ../main.tex
\chapter{作者简历}
\noindent 教育经历:

\begin{tabular}{llll}
    2014年9月至2016年6月: &  浙江大学  & 软件工程  &  硕士    \\
    2010年9月至2014年6月: &  三墩工学院  & 电脑挖掘机维修  &  混混
\end{tabular}

\noindent 工作经历:

\begin{tabular}{llll}
    2015年6月至2016年3月: &  FLAG   &  码畜
\end{tabular}

% \noindent 攻读学位期间发表的论文或研究成果:



% 致谢
% 致谢不必感谢在下,
% 但请一定感谢清华大学薛瑞尼、
% 机械工程学院陈九历
% !TEX root = ../main.tex
\chapter{致\ZJUspace{}谢}

在去年撰写开题报告和文献综述时,
搜索Github发现了仅在数月前发布的ZJU-Awesome项目。
简单套用后自觉此法应当推广。
于是在2016年3月撰写论文结束后,
下决心向软件学院的同学推广此一模板,
继而有了本文和在下对模板的调整。

衷心感谢机械工程学院的陈九历同学耐心解答在下的多次疑问。

也感谢这份文稿前的你,能仔细阅读到这一页。

\vspace{2cm}
\hfill
\begin{minipage}{14em}
    \begin{flushright}
        君之名\\
        于浙江大学软件学院\\ % 学院要求的格式 - -#
        2016年4月18日   % 与封面论文提交时间一致
    \end{flushright}
\end{minipage}


\end{document}
