% !Mode:: "TeX:UTF-8"
% !TEX root = ..\main.tex
\chapter{致\ZJUspace 谢}
岁月如梭,转眼间,X年的XX生求学生活即将结束,站在毕业的门槛上,回首往昔,奋斗和辛劳成为丝丝的记忆,甜美与欢笑也都尘埃落定。浙江大学以其优良的学习风气、严谨的科研氛围教我求学,以其博大包容的情怀胸襟、浪漫充实的校园生活育我成人。值此毕业论文完成之际,我谨向所有关心、爱护、帮助我的人们表示最诚挚的感谢与最美好的祝愿。

本论文是在导师诸葛亮丞相的悉心指导之下完成的。三年来,导师渊博的专业知识,严谨的治学态度,精益求精的工作作风,诲人不倦的高尚师德,朴实无华、平易近人的人格魅力对我影响深远。导师不仅授我以文,而且教我做人,虽历时X载,却赋予我终生受益无穷之道。本论文从选题到完成,几易其稿,每一步都是在导师的指导下完成的,倾注了导师大量的心血,在此我向我的导师诸葛丞相表示深切的谢意与祝福!

本论文的完成也离不开其他各位老师、同学和朋友的关心与帮助。在此也要感谢关羽、赵云等各位老师在论文开题、初稿、预答辩期间所提出的宝贵意见,感谢木牛流马课题组为本论文提供的数据和建议,还要感谢同门的师兄师妹们,在科研过程中给我以许多鼓励和帮助。回想整个论文的写作过程,虽有不易,却让我除却浮躁,经历了思考和启示,也更加深切地体会了法学的精髓和意义,因此倍感珍惜。

还要感谢父母在我求学生涯中给与我无微不至的关怀和照顾,一如既往地支持我、鼓励我。同时,还要感谢关某某同学、赵某某同学、张某某同学、黄某某同学X年来对我的爱护、包容和帮助,愿友谊长存!

\vspace{2cm}
\hfill
\begin{minipage}{14em}
\begin{center}
于XX地\quad 225年1月1日\\
我的名
\end{center}
\end{minipage}
